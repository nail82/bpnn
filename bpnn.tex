\documentclass{article}
\usepackage{amsmath, amsthm, amssymb}
\usepackage{fancyhdr}
\usepackage{setspace}
\usepackage[dvips]{graphicx}
\usepackage{ifthen}
\usepackage{lastpage}
\usepackage{extramarks}
\usepackage{upgreek}
\usepackage{listings}
\usepackage[left=0.75in, top=1in, right=0.75in, bottom=1in]{geometry}

\lstloadlanguages{Python}
\lstset{language=Python,
        frame=single,
       }

% Homework Specific Information
\newcommand{\hmwkTitle}{Assignment 3}
\newcommand{\hmwkSubTitle}{}
\newcommand{\hmwkDueDate}{12 October 2012}
\newcommand{\hmwkClass}{CS 640}
\newcommand{\hmwkClassTime}{2:20}
\newcommand{\hmwkClassInstructor}{Dr Ranganath}
\newcommand{\hmwkAuthorName}{Ted Satcher}

\pagestyle{fancy}
\lhead{\hmwkAuthorName}
\chead{\hmwkClass\ (\hmwkClassInstructor\ \hmwkClassTime): \hmwkTitle}
\rhead{\firstxmark}
\lfoot{\lastxmark}
\cfoot{}
\rfoot{Page\ \thepage\ of\ \protect\pageref{LastPage}}
\renewcommand\headrulewidth{0.4pt}
\renewcommand\footrulewidth{0.4pt}

\title{\vspace{2in}\textmd{\textbf{\hmwkClass:\ \hmwkTitle\ifthenelse{\equal{\hmwkSubTitle}{}}{}{\\\hmwkSubTitle}}}\\\normalsize\vspace{0.1in}\small{Due\ on\ \hmwkDueDate}\\\vspace{0.1in}\large{\textit{\hmwkClassInstructor\ \hmwkClassTime}}\vspace{3in}}
\date{}
\author{\textbf{\hmwkAuthorName}}

\begin{document}
\maketitle

\section*{Introduction}

\section*{XOR Problem}
- Set the problem up with the same values in the text.
- Iteration limit of 1000 wouldn't converge, in fact, I didn't think
it was converging.  Increased the limit and plotted the error and
noticed that on some runs it was trending to a convergence.

- Turned off the iteration limit and allowed to run until error was
$\lt 0.01$.  It took 80705 iterations and from plot, error is slowly
descreasing.  Clear from the result that an arbitrary error level can
be reached if you want to wait long enough.

\section*{NOTES}
- Starting work on the neural net.

- Coming to the conclusion that a big part of neural network
design is how to set up the data and get access to it.

- My choice in this assignment was to restrict the problem to a
three layer network, input, hidden and output, but to not
otherwise restrict it.  That is, I wanted the ability to add
an arbitray number of input, hidded and output nodes.  I also
wanted the flexibility to swap out the squashing function.

- The primary data structure for my network is in the DataTable
class in the data_table module.  The DataTable class is the central
element of my architecture.  Its primary purpose is to provide
access to the data elements representing the network to clients
which are primarily the nodes in the network.

- Book keeping seem is a major task putting one of these together.

- xor problem isn't converging.  error is just oscillating.  tried
2,3,4 hidden nodes with similar results

- serializing the network is important.
- parameterized the network to accept variable numbers of input,
hidden and output nodes.

- architecture
  - data table to hold weights and intermediate calculations
  table doesn't do any operations on the data other than storing and
  retrieving data

- nodes are objects (but this should change.



\end{document}